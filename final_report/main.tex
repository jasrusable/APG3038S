\documentclass{article}

\usepackage{graphicx}
\usepackage{amsmath}
\usepackage[backend=bibtex]{biblatex}
\addbibresource{sources.bib}

\begin{document}

\begin{titlepage}
	\begin{center}
		\vspace*{1cm}
		
		\textbf{Report on proposed development plans for Flamingoville}
		
		\vspace{1.5cm}
		
		\textbf{Prepared for:}
		
		Mr Jacob de Wet
		
		Flamingoville Town Council
		
		\vspace{1.5cm}
		
		\textbf{Prepared by:}
		
		Mr Jason Russell
		
		Environmentally-aware expert
		
		\vspace{1.5cm}

		\textbf{Key Words:}
		Development, expansion, feasibilty
		
		\vspace{1.5cm}
		September 2015
		
	\end{center}
\end{titlepage}
\thispagestyle{empty}

\pagenumbering{roman}
\setcounter{page}{0}

\newpage
\section*{Terms of reference}
In September 2015, Mr Jacob de Wet, a member of the Flamingoville Town Council, initiated this development. The need for it arose as Flamingoville is currenlt unable to accommodate a large influx of annual visitors and a growing labour force of nearby mines. The Council is forced to develop in the laggon area as there is no other suitable land avaliable.

\paragraph{}

Mr Jacob's specific instructions were the following:

\begin{enumerate}
	\item Examine three propsed development plans given to the Town Council.
	\item Gather information about the existence and location of birds, flora and fauna in the area.
	\item For each plan:
	\begin{enumerate}
		\item Describe the effect of the development as a whole on the marine life of the lagoon itself and its tributary streams and beaches, and describe the counter measures suggested.
		\item Describe the economic and visual impact (aesthetics) of the development on the town and the nature reserve, incorporating the point of view of Flamingoville town residents.
	\end{enumerate}
	\item Draw conclusions on the basis of all the findings and recommend the plan that will do the least damage, and that will have the most positive long term impact on the area.
\end{enumerate}

\newpage
\section*{Executive Summary}

\newpage
\tableofcontents

\pagenumbering{arabic}
\setcounter{page}{0}

\newpage
\section{INTRODUCTION}
\subsection{Subject of and motivation for report}
This report concerns the further development of Flamingoville. Three development plans have been proposed, each will be detailed and discussed in this report.

\subsection{Background to investigation}
The Flamingoville Town Council needs to extend the town. The extension is needed in order to accommodate the large influx of annual visitors as well as the growing labour force which has been attracted to Falmingoville because of the mining in the area. Due to the lack of suitable developable land, the Council is forced to develop in the lagoon area.

\subsection{Objectives of report}
The objectives of this report are therefor to:
\begin{itemize}
	\item examine the three proposed development plans
	\item gather information about the existence and location of birds, flore and fauna in the development area
	\item describe the effect of development on marine life of the lagoon and describe counter measures
	\item describe the economic impact of development on te town and nature reserve
	\item draw conclusions based on the findings
	\item make reccommendations as to which development plan to use, based on least damge with most positive long term impact on area
\end{itemize}

\newpage
\section{FINDINGSSS (Dont actually mention 'findings')}

\newpage
\section{CONCLUSIONS}

\newpage
\section{RECOMMENDATIONS}

\newpage
\printbibliography

\end{document}